\documentclass[12pt,,]{report}
\usepackage{lmodern}
\usepackage{amssymb,amsmath}
\usepackage{ifxetex,ifluatex}
\usepackage{fixltx2e} % provides \textsubscript
\ifnum 0\ifxetex 1\fi\ifluatex 1\fi=0 % if pdftex
  \usepackage[T1]{fontenc}
  \usepackage[utf8]{inputenc}
\else % if luatex or xelatex
  \ifxetex
    \usepackage{mathspec}
  \else
    \usepackage{fontspec}
  \fi
  \defaultfontfeatures{Ligatures=TeX,Scale=MatchLowercase}

    \usepackage{xeCJK}
    % 中文自動換行
    \XeTeXlinebreaklocale "zh"
    % 文字的彈性間距
    \XeTeXlinebreakskip = 0pt plus 1pt
    \newfontlanguage{Chinese}{CHN}
    % 章次20級,節次16級,小節次以下14級,本文12級字
    \def\LARGE{\fontsize{20}{30}\selectfont}%章次
    \def\Large{\fontsize{16}{24}\selectfont}%節次
    \def\large{\fontsize{14}{21}\selectfont}%小節次
    \usepackage{indentfirst}
    \usepackage{CJKnumb}
    \renewcommand{\figurename}{圖}
    \renewcommand{\thefigure}{{\arabic{chapter}}.\arabic{figure}}
    \renewcommand{\tablename}{表}
    \renewcommand{\thetable}{{\arabic{chapter}}.\arabic{table}}
    %重製章節
    \renewcommand{\chaptername}{}
    \renewcommand{\thechapter}{第\CJKnumber{\arabic{chapter}}章}
    \renewcommand{\thesection}{{\arabic{chapter}}.\arabic{section}}
    \renewcommand{\thesubsection}{{\arabic{chapter}}.{\arabic{section}}.\arabic{subsection}}
    %設定行距與中英文字型
    \linespread{1}\selectfont
    \setCJKmainfont{SimSun}
    \setmainfont{Times New Roman}
    \setromanfont{Times New Roman}
    \setmonofont{Times New Roman}
    %重製章節標籤
    \usepackage{titlesec}
    \titleformat{\chapter}[block]{\LARGE\centering}{\thechapter}{0.5em}{}
    \titleformat{\section}[block]{\Large}{\thesection}{0.5em}{}
    \titleformat{\subsection}[block]{\large}{\thesubsection}{0.5em}{}
    % 重製目錄
    \usepackage{titletoc}
    \titlespacing{\chapter}{0pt}{*0}{*2}
    \titlespacing{\section}{0pt}{*1}{*1}
    \titlespacing{\subsection}{0pt}{*1}{*1}
    \titlespacing{\subsubsection}{0pt}{*1}{*1}
    \titlecontents{chapter}[0em]{}{\contentspush{\thecontentslabel}\hspace*{1em}}{}{\titlerule*[0.7pc]{.}\contentspage}
\fi
% use upquote if available, for straight quotes in verbatim environments
\IfFileExists{upquote.sty}{\usepackage{upquote}}{}
% use microtype if available
\IfFileExists{microtype.sty}{
\usepackage{microtype}
\UseMicrotypeSet[protrusion]{basicmath} % disable protrusion for tt fonts
}{}
\usepackage[margin=1in]{geometry}
\usepackage[unicode=true]{hyperref}
\hypersetup{
            pdfauthor={設計一乙 40623204 張晏晴; 設計一乙 40623206 邱妍蓁; 設計一乙 40623214 吳家文; 設計一乙 40623215 吳柏毅; 設計一乙 40623249 李秉樺; 設計一乙 40623251 鄧靖宣},
            pdfborder={0 0 0},
            breaklinks=true}
\urlstyle{same}  % don't use monospace font for urls
\ifnum 0\ifxetex 1\fi\ifluatex 1\fi=0 % if pdftex
  \usepackage[shorthands=off,main=]{babel}
\else
  \usepackage{polyglossia}
  \setmainlanguage[]{}
\fi
\usepackage{longtable,booktabs}
% Fix footnotes in tables (requires footnote package)
\IfFileExists{footnote.sty}{\usepackage{footnote}\makesavenoteenv{long table}}{}
\IfFileExists{parskip.sty}{%
\usepackage{parskip}
}{% else
\setlength{\parindent}{0pt}
\setlength{\parskip}{6pt plus 2pt minus 1pt}
}
\setlength{\emergencystretch}{3em}  % prevent overfull lines
\providecommand{\tightlist}{%
  \setlength{\itemsep}{0pt}\setlength{\parskip}{0pt}}
\setcounter{secnumdepth}{5}
% Redefines (sub)paragraphs to behave more like sections
\ifx\paragraph\undefined\else
\let\oldparagraph\paragraph
\renewcommand{\paragraph}[1]{\oldparagraph{#1}\mbox{}}
\fi
\ifx\subparagraph\undefined\else
\let\oldsubparagraph\subparagraph
\renewcommand{\subparagraph}[1]{\oldsubparagraph{#1}\mbox{}}
\fi

% set default figure placement to htbp
\makeatletter
\def\fps@figure{htbp}
\makeatother


\begin{document}
%Cover Start
\begin{titlepage}
\vspace{1cm}
\begin{center}
\fontsize{36}{54}\selectfont{
    國立虎尾科技大學\par
}
\fontsize{28}{42}\selectfont{機械設計工程系\par}
\fontsize{24}{36}\selectfont{計算機程式 bg2 期末報告\par}
\vspace{1.5cm}
\fontsize{20}{30}\selectfont{
    PyQt5 事件導向計算器\par
    PyQt5 Event-Driven Calculator Project\par
}
\vspace{\fill}
\fontsize{18}{27}\selectfont{
    學生:\par
    設計一乙 40623204 張晏晴 \par 設計一乙 40623206 邱妍蓁 \par 設計一乙 40623214 吳家文 \par 設計一乙 40623215 吳柏毅 \par 設計一乙 40623249 李秉樺 \par 設計一乙 40623251 鄧靖宣 \par
    指導教授:嚴家銘\par
}
\vspace{1.5cm}
\fontsize{16}{24}\selectfont{2017.12.18\par}
\end{center}
\vspace{1cm}
\end{titlepage}

\newcommand\frontmatter{
    \cleardoublepage
    \pagenumbering{roman}
}

\newcommand\mainmatter{
    \cleardoublepage
    \pagenumbering{arabic}
}

\newcommand\backmatter{
    \if@openright
        \cleardoublepage
    \else
        \clearpage
    \fi
}

%Document start

% Set page number to arabic i ii...
\frontmatter
\renewcommand{\abstractname}{\LARGE \center 摘要}
\chapter*{摘要}
\addcontentsline{toc}{chapter}{摘要}
\fontsize{14}{21}\selectfont{這裡是摘要內容。A pipe character, followed by an indented block of text
is treated as a literal block, in which newlines are preserved
throughout the block, including the final newline.

\begin{itemize}
\tightlist
\item
  以 YAML 的方式插入。
\item
  The `+' indicator says to keep newlines at the end of text blocks.
\item
  使用 Markdown 語法。
\item
  前面使用加號
\end{itemize}

本研究的重點在於 \ldots{}}


\begingroup
    \renewcommand{\contentsname}{\center 目錄 \addcontentsline{toc}{chapter}{目錄}}
    \renewcommand{\numberline}[1]{~#1\hspace*{1em}}
        \setcounter{tocdepth}{2}
    \tableofcontents
    \newcommand{\lotlabel}{表}
    \renewcommand{\listtablename}{\center 表目錄 \addcontentsline{toc}{chapter}{表目錄}}
    \renewcommand{\numberline}[1]{\lotlabel~#1\hspace*{1em}}
    \listoftables
    \newcommand{\loflabel}{圖}
    \renewcommand{\listfigurename}{\center 圖目錄 \addcontentsline{toc}{chapter}{圖目錄}}
    \renewcommand{\numberline}[1]{\loflabel~#1\hspace*{1em}}
    \listoffigures
\endgroup

% Start normal page number, 1 2 3
\mainmatter
\hypertarget{ux524dux8a00}{%
\chapter{前言}\label{ux524dux8a00}}

計算器程式期末報告前言

前言內容。

~

一個範例數學式:\[\beta=\cos^{-1}{\frac{L0^{2}+d_{AB}^{2}-R0^{2}}{2\times{L0\times{d_{AB}}}}}\]

~

關於數學式可以參考這裡:\url{http://www.hostmath.com/}

提及了某篇刊物{[}1{]}在這裡。

\hypertarget{ux53efux651cux7a0bux5f0fux7cfbux7d71ux4ecbux7d39}{%
\chapter{可攜程式系統介紹}\label{ux53efux651cux7a0bux5f0fux7cfbux7d71ux4ecbux7d39}}

可攜程式系統介紹

\hypertarget{ux555fux52d5ux8207ux95dcux9589}{%
\section{啟動與關閉}\label{ux555fux52d5ux8207ux95dcux9589}}

Windows 的內容

有一張圖片:

{[}Kmol{]}{[}{]}

稱為圖 \{\textbf{???}\}。

各 md 檔案可以在 images 目錄下自訂與 md 檔案名稱相同的子目錄存放影像檔案

\hypertarget{ux555fux52d5ux8207ux95dcux95892}{%
\section{啟動與關閉2}\label{ux555fux52d5ux8207ux95dcux95892}}

\hypertarget{python-ux7a0bux5f0fux8a9eux6cd5}{%
\chapter{Python 程式語法}\label{python-ux7a0bux5f0fux8a9eux6cd5}}

Python 程式語法

變數命名

變數必須以英文字母大寫或小寫或底線開頭
變其餘字元可以是英文大小寫字母、數字、底線 變數區分英文大小寫
變數不限字元長度 不可使用關鍵字當變數名稱 Python3 的程式關鍵字,
使用者命名變數時, 必須避開下列保留字.
{[}``False'',''None'',''True'',''and'',''as'',''assert'',''break'',
''class'',''continue'',''def'',''del'',''elif'',''else'',''except'',
``finally'',''for'',''from'',''global'',''if'',''import'',''in'',
''is'',''lambda'',''nonlocal'',''not'',''or'',''pass'',''raise'',
``return'',''try'',''while'',''with'',''yield''{]}

print 函式

print() 為 Python 程式語言中用來列印數值或字串的函式, 其中有 sep
變數定義分隔符號, sep 內定為 ``,'', end 變數則用來定義列印結尾的符號,
end 內定為跳行符號。

重複迴圈

用於已知重複次數的程式,迴圈結構中指定指定迴圈變數的初始值、終止值語遞增(減)值,迴圈變數將由初始值變化到終止值的前一個數字每次依照遞增(減)的值進行數值遞增或遞減。

判斷式

If self.display.text() = = `0'and digitValue = = 0.0:

Return

= =判斷式的等於要兩個

' 0 '表示字串

and有時候用\textbar{}\textbar{}

:還沒說完

數列

Digits = {[}self one , \ldots{}{]}

{[}{]}表示數列

\表示還沒講完

\hypertarget{pyqt5-ux7c21ux4ecb}{%
\chapter{PyQt5 簡介}\label{pyqt5-ux7c21ux4ecb}}

說明 PyQt5 基本架構與程式開發流程

\hypertarget{pyqt5-ux67b6ux69cb}{%
\section{PyQt5 架構}\label{pyqt5-ux67b6ux69cb}}

C 的內容

其中包含一個表格:

\begin{longtable}[]{@{}ccccccccc@{}}
\caption{Python 網際框架比較 \label{tbl:網際框架}}\tabularnewline
\toprule
\begin{minipage}[b]{0.09\columnwidth}\centering
Framework\strut
\end{minipage} & \begin{minipage}[b]{0.07\columnwidth}\centering
Started\strut
\end{minipage} & \begin{minipage}[b]{0.04\columnwidth}\centering
Py2\strut
\end{minipage} & \begin{minipage}[b]{0.04\columnwidth}\centering
Py3\strut
\end{minipage} & \begin{minipage}[b]{0.04\columnwidth}\centering
ORM\strut
\end{minipage} & \begin{minipage}[b]{0.13\columnwidth}\centering
Template Engine\strut
\end{minipage} & \begin{minipage}[b]{0.11\columnwidth}\centering
Auth Moudule\strut
\end{minipage} & \begin{minipage}[b]{0.12\columnwidth}\centering
Database Admin\strut
\end{minipage} & \begin{minipage}[b]{0.11\columnwidth}\centering
Project Scale\strut
\end{minipage}\tabularnewline
\midrule
\endfirsthead
\toprule
\begin{minipage}[b]{0.09\columnwidth}\centering
Framework\strut
\end{minipage} & \begin{minipage}[b]{0.07\columnwidth}\centering
Started\strut
\end{minipage} & \begin{minipage}[b]{0.04\columnwidth}\centering
Py2\strut
\end{minipage} & \begin{minipage}[b]{0.04\columnwidth}\centering
Py3\strut
\end{minipage} & \begin{minipage}[b]{0.04\columnwidth}\centering
ORM\strut
\end{minipage} & \begin{minipage}[b]{0.13\columnwidth}\centering
Template Engine\strut
\end{minipage} & \begin{minipage}[b]{0.11\columnwidth}\centering
Auth Moudule\strut
\end{minipage} & \begin{minipage}[b]{0.12\columnwidth}\centering
Database Admin\strut
\end{minipage} & \begin{minipage}[b]{0.11\columnwidth}\centering
Project Scale\strut
\end{minipage}\tabularnewline
\midrule
\endhead
\begin{minipage}[t]{0.09\columnwidth}\centering
Pyramid\strut
\end{minipage} & \begin{minipage}[t]{0.07\columnwidth}\centering
2005\strut
\end{minipage} & \begin{minipage}[t]{0.04\columnwidth}\centering
V\strut
\end{minipage} & \begin{minipage}[t]{0.04\columnwidth}\centering
V\strut
\end{minipage} & \begin{minipage}[t]{0.04\columnwidth}\centering
\strut
\end{minipage} & \begin{minipage}[t]{0.13\columnwidth}\centering
\strut
\end{minipage} & \begin{minipage}[t]{0.11\columnwidth}\centering
V\strut
\end{minipage} & \begin{minipage}[t]{0.12\columnwidth}\centering
\strut
\end{minipage} & \begin{minipage}[t]{0.11\columnwidth}\centering
large\strut
\end{minipage}\tabularnewline
\begin{minipage}[t]{0.09\columnwidth}\centering
Django\strut
\end{minipage} & \begin{minipage}[t]{0.07\columnwidth}\centering
2006\strut
\end{minipage} & \begin{minipage}[t]{0.04\columnwidth}\centering
V\strut
\end{minipage} & \begin{minipage}[t]{0.04\columnwidth}\centering
V\strut
\end{minipage} & \begin{minipage}[t]{0.04\columnwidth}\centering
V\strut
\end{minipage} & \begin{minipage}[t]{0.13\columnwidth}\centering
V\strut
\end{minipage} & \begin{minipage}[t]{0.11\columnwidth}\centering
V\strut
\end{minipage} & \begin{minipage}[t]{0.12\columnwidth}\centering
V\strut
\end{minipage} & \begin{minipage}[t]{0.11\columnwidth}\centering
large\strut
\end{minipage}\tabularnewline
\begin{minipage}[t]{0.09\columnwidth}\centering
Flask\strut
\end{minipage} & \begin{minipage}[t]{0.07\columnwidth}\centering
2010\strut
\end{minipage} & \begin{minipage}[t]{0.04\columnwidth}\centering
V\strut
\end{minipage} & \begin{minipage}[t]{0.04\columnwidth}\centering
\strut
\end{minipage} & \begin{minipage}[t]{0.04\columnwidth}\centering
\strut
\end{minipage} & \begin{minipage}[t]{0.13\columnwidth}\centering
\strut
\end{minipage} & \begin{minipage}[t]{0.11\columnwidth}\centering
\strut
\end{minipage} & \begin{minipage}[t]{0.12\columnwidth}\centering
\strut
\end{minipage} & \begin{minipage}[t]{0.11\columnwidth}\centering
small\strut
\end{minipage}\tabularnewline
\bottomrule
\end{longtable}

稱為表 \ref{tbl:網際框架}。

\begin{longtable}[]{@{}lcr@{}}
\caption{價目表 \label{tbl:價目表}}\tabularnewline
\toprule
Tables & Are & Cool\tabularnewline
\midrule
\endfirsthead
\toprule
Tables & Are & Cool\tabularnewline
\midrule
\endhead
col 1 is & left-aligned & \$1600\tabularnewline
col 2 is & centered & \$12\tabularnewline
col 3 is & right-aligned & \$1\tabularnewline
\bottomrule
\end{longtable}

稱為表 \ref{tbl:價目表}。

關於表格生成可以參考這裡:\url{http://www.tablesgenerator.com/markdown_tables}

\hypertarget{calculator-ux7a0bux5f0f}{%
\chapter{Calculator 程式}\label{calculator-ux7a0bux5f0f}}

Calculator 程式細部說明

\hypertarget{ux5efaux7acbux5c0dux8a71ux6846}{%
\section{建立對話框}\label{ux5efaux7acbux5c0dux8a71ux6846}}

自動控制 的內容

其中包含一個表格:

\begin{longtable}[]{@{}ccccccccc@{}}
\caption{Python 網際框架比較 \label{tbl:網際框架}}\tabularnewline
\toprule
\begin{minipage}[b]{0.09\columnwidth}\centering
Framework\strut
\end{minipage} & \begin{minipage}[b]{0.07\columnwidth}\centering
Started\strut
\end{minipage} & \begin{minipage}[b]{0.04\columnwidth}\centering
Py2\strut
\end{minipage} & \begin{minipage}[b]{0.04\columnwidth}\centering
Py3\strut
\end{minipage} & \begin{minipage}[b]{0.04\columnwidth}\centering
ORM\strut
\end{minipage} & \begin{minipage}[b]{0.13\columnwidth}\centering
Template Engine\strut
\end{minipage} & \begin{minipage}[b]{0.11\columnwidth}\centering
Auth Moudule\strut
\end{minipage} & \begin{minipage}[b]{0.12\columnwidth}\centering
Database Admin\strut
\end{minipage} & \begin{minipage}[b]{0.11\columnwidth}\centering
Project Scale\strut
\end{minipage}\tabularnewline
\midrule
\endfirsthead
\toprule
\begin{minipage}[b]{0.09\columnwidth}\centering
Framework\strut
\end{minipage} & \begin{minipage}[b]{0.07\columnwidth}\centering
Started\strut
\end{minipage} & \begin{minipage}[b]{0.04\columnwidth}\centering
Py2\strut
\end{minipage} & \begin{minipage}[b]{0.04\columnwidth}\centering
Py3\strut
\end{minipage} & \begin{minipage}[b]{0.04\columnwidth}\centering
ORM\strut
\end{minipage} & \begin{minipage}[b]{0.13\columnwidth}\centering
Template Engine\strut
\end{minipage} & \begin{minipage}[b]{0.11\columnwidth}\centering
Auth Moudule\strut
\end{minipage} & \begin{minipage}[b]{0.12\columnwidth}\centering
Database Admin\strut
\end{minipage} & \begin{minipage}[b]{0.11\columnwidth}\centering
Project Scale\strut
\end{minipage}\tabularnewline
\midrule
\endhead
\begin{minipage}[t]{0.09\columnwidth}\centering
Pyramid\strut
\end{minipage} & \begin{minipage}[t]{0.07\columnwidth}\centering
2005\strut
\end{minipage} & \begin{minipage}[t]{0.04\columnwidth}\centering
V\strut
\end{minipage} & \begin{minipage}[t]{0.04\columnwidth}\centering
V\strut
\end{minipage} & \begin{minipage}[t]{0.04\columnwidth}\centering
\strut
\end{minipage} & \begin{minipage}[t]{0.13\columnwidth}\centering
\strut
\end{minipage} & \begin{minipage}[t]{0.11\columnwidth}\centering
V\strut
\end{minipage} & \begin{minipage}[t]{0.12\columnwidth}\centering
\strut
\end{minipage} & \begin{minipage}[t]{0.11\columnwidth}\centering
large\strut
\end{minipage}\tabularnewline
\begin{minipage}[t]{0.09\columnwidth}\centering
Django\strut
\end{minipage} & \begin{minipage}[t]{0.07\columnwidth}\centering
2006\strut
\end{minipage} & \begin{minipage}[t]{0.04\columnwidth}\centering
V\strut
\end{minipage} & \begin{minipage}[t]{0.04\columnwidth}\centering
V\strut
\end{minipage} & \begin{minipage}[t]{0.04\columnwidth}\centering
V\strut
\end{minipage} & \begin{minipage}[t]{0.13\columnwidth}\centering
V\strut
\end{minipage} & \begin{minipage}[t]{0.11\columnwidth}\centering
V\strut
\end{minipage} & \begin{minipage}[t]{0.12\columnwidth}\centering
V\strut
\end{minipage} & \begin{minipage}[t]{0.11\columnwidth}\centering
large\strut
\end{minipage}\tabularnewline
\begin{minipage}[t]{0.09\columnwidth}\centering
Flask\strut
\end{minipage} & \begin{minipage}[t]{0.07\columnwidth}\centering
2010\strut
\end{minipage} & \begin{minipage}[t]{0.04\columnwidth}\centering
V\strut
\end{minipage} & \begin{minipage}[t]{0.04\columnwidth}\centering
\strut
\end{minipage} & \begin{minipage}[t]{0.04\columnwidth}\centering
\strut
\end{minipage} & \begin{minipage}[t]{0.13\columnwidth}\centering
\strut
\end{minipage} & \begin{minipage}[t]{0.11\columnwidth}\centering
\strut
\end{minipage} & \begin{minipage}[t]{0.12\columnwidth}\centering
\strut
\end{minipage} & \begin{minipage}[t]{0.11\columnwidth}\centering
small\strut
\end{minipage}\tabularnewline
\bottomrule
\end{longtable}

稱為表 \ref{tbl:網際框架}。

\begin{longtable}[]{@{}lcr@{}}
\caption{價目表 \label{tbl:價目表}}\tabularnewline
\toprule
Tables & Are & Cool\tabularnewline
\midrule
\endfirsthead
\toprule
Tables & Are & Cool\tabularnewline
\midrule
\endhead
col 1 is & left-aligned & \$1600\tabularnewline
col 2 is & centered & \$12\tabularnewline
col 3 is & right-aligned & \$1\tabularnewline
\bottomrule
\end{longtable}

稱為表 \ref{tbl:價目表}。

關於表格生成可以參考這裡:\url{http://www.tablesgenerator.com/markdown_tables}

\hypertarget{ux5efaux7acbux6309ux9215}{%
\section{建立按鈕}\label{ux5efaux7acbux6309ux9215}}

從左邊工具列拉按鈕``Push Button'' {[}P-Button{]}{[}{]}

稱為圖 \{\textbf{???}\}。

並更改每一項的命名``objectName'' {[}P-Button{]}: ./images/P-Button.PNG
\{\#fig:P-Button\}

以及拉顯示欄``Line Edit'' {[}P-Display{]}{[}{]}

稱為圖 \{\textbf{???}\}。

更改其名稱為``display'' {[}P-Display{]}: ./images/P-Display.png
\{\#fig:P-Display\}

\hypertarget{ux5efaux7acbux7a0bux5f0fux78bc}{%
\section{建立程式碼}\label{ux5efaux7acbux7a0bux5f0fux78bc}}

導入 sys 模組→import sys

從PyQt5模組導入QtWidgets模組→from PyQt5 import QtWidgets

從ui目錄導入Dialog.py, Python 程式檔案本身就是一個模組,
且文件名就是模組名→from ui import Dialog

定義各數字名稱→number = {[}self.one, self.two, self.three, self.four ,
self.five, self.six, self.seven, self.eight, self.nine, self.zero{]}

以及符號名稱→plus\_minus = {[}self.plusButton, self.minusButton{]}

times\_division = {[}self.timesButton, self.divisionButton{]}

unary = {[}self.reciprocalButton, self.squareRootButton,
self.powerButton{]}

建立迴圈使程式簡單化

{[}for O{]}{[}{]}

稱為圖 \{\textbf{???} O\}。 {[}for O{]}: ./images/for O.png \{\#fig:for
O\}

其餘建立各個Button的函式

\hypertarget{ux5fc3ux5f97}{%
\chapter{心得}\label{ux5fc3ux5f97}}

期末報告心得

\hypertarget{fossil-scm}{%
\section{Fossil SCM}\label{fossil-scm}}

\hypertarget{ux7db2ux8a8cux5fc3ux5f97}{%
\section{網誌心得}\label{ux7db2ux8a8cux5fc3ux5f97}}

Github 協同倉儲

bg2:https://github.com/40623206/bg2\_pyqt5\_calculator

說明各學員任務與執行過程

40623204-

40623206

數字、數字疊加、+、-、=、計算、中斷運算、clearAll

因為waitingForOperand的waiting少了ing,所以一直無法順利執行

+跟-的部分用了For迴圈及各自跟下面做連結,導致+按下去\emph{2,-按下去}0

40623214-

40623215

雖然看老師的函式都能了解其意思,但自己要憑空打出來卻比想像中困難。

40623249-

40623251-

學員心得

40623204-

40623206

從一開始的eric6建立Dialog拉按鈕及顯示幕到使用物件導向編寫程式,用運老師所教的,利用For迴圈連結同屬性的東西,才不會使程式複雜化。

40623214-

40623215-

40623249-

40623251-

\hypertarget{ux7d50ux8ad6}{%
\chapter{結論}\label{ux7d50ux8ad6}}

期末報告結論

\hypertarget{ux7d50ux8ad6ux8207ux5efaux8b70}{%
\section{結論與建議}\label{ux7d50ux8ad6ux8207ux5efaux8b70}}

結論與建議內容

\hypertarget{ux53c3ux8003ux6587ux737b}{%
\chapter*{參考文獻}\label{ux53c3ux8003ux6587ux737b}}
\addcontentsline{toc}{chapter}{參考文獻}

\hypertarget{refs}{}
\leavevmode\hypertarget{ref-myart}{}%
{[}1{]} 作者名字, ``標題,'' \emph{刊物名稱}, vol. 4, no. 2, pp.
201--213, Jul. 1993.


\end{document}
